\section{System Design and Architecture}
This chapter represents the main body of your project. It should describe the project in full details. This chapter should answer the questions: “what has been done?” and “how it has been done?”. As such, the steps you went through to realize the project should be highlighted and properly discussed. Your scientific approaches and methodologies should be clarified. The discussion should adopt a logical flow starting from the whole block diagram, to coarse modules, and finally to fine modules. While writing this chapter, try to give as much details as possible, such that an interested reader could easily replicate your work and improve it.

In this space, before the first section, write an introductory paragraph on how you design and build your project


\subsection{Overview and Assumptions}In this section, introduce how you design you system and develop its underlying architecture. Any employed assumptions should be clearly enumerated and justified.

\subsection{System Architecture}
The architecture of your system should be given in this section. This architecture should be first represented as a block diagram (subsection 5.2.1), which clarifies different project modules and the connections between them. You may add more subsections to properly explain your design. If possible, flowcharts are better included to ensure that the big picture and the interaction between different modules are very clear to the reader. Thereafter, each module should have a separate subsequent section to clearly describe and discuss it.

\subsection{Block Diagram}
Draw the block diagram of your architecture and generally discuss its modules. After reading this subsection, interested audience should have understood the big picture of your system design and architecture. The interaction between modules should also be conveyed in this subsection

\subsection{Module 1}
Each module within your architecture should have a distinct section to explain the design of the module itself. Again, give as much details as possible, so that the reader could easily understand how the module is designed and what are the constraints that affect its design?

\subsubsection{Functional Description}
\subsubsection{Modular Decomposition}
\subsubsection{Design Constraints}
\subsubsection{Other Description of Module 1}